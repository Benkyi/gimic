\documentclass[a4paper,11pt]{article}
\usepackage[english]{babel}
\usepackage[T1]{fontenc}
\usepackage[utf8]{inputenc}
%\usepackage{hyperref}
\usepackage{times}
\usepackage{mystuff}
\usepackage{mytitlepage}
%\usepackage[pdftex,colorlinks=true,
%   pdfstartview=FitV,
%   linkcolor=blue,
%   citecolor=blue,
%   urlcolor=blue]{hyperref}

\begin{document}
\title{GIMIC 2.0 users manual}
\subtitle{Gauge-Including Magnetically Induced Currents}

%\subtitle{User manual}
\author{{\sf Jonas Jus\'elius}}
\address{
{\sf University of Tromsø}\\
{\sf Department of Chemistry}\\
{\sf N-9037 Tromsø}
}
\years{2004-2007}
\maketitle

\section{Introduction}
This is the GIMIC  program for calculating magnetically induced currents in
molecules. For this program produce any kind of useful information, you 
need to provide it with an AO density matrix and three (effective)
magnetically perturbed AO density matrices in the proper format. Currently
only recent versions of ACES2 and Turbomole can produce these matrices, but
Dalton is in the works. If you would like to add your favourite program to the
list please use the source, Luke.

\begin{itemize}
  \item For instructions how to compile and install this program refer to
  the INSTALL file in the top level directory. 
\item For more detailed information on how to use the program read the documents 
  found in the Documentation directory. Note that the manual is obsolete.
\item There is an annotated example input in the examples/ directory. 
\item For information on command line flags available run: 
  '\texttt{gimic --help}'
\end{itemize}

The following features have been implemented in the
program
\begin{itemize}
  \item Current densities in 2D or 3D
  \item The modulus of the current 	
  \item The divergence of the current (this is useful for checking gauge
	invariance vs. gauge independence)
  \item Vector representation of the current in 2D or 3D
  \item Integration of the current flow through defined cut-planes in
	molecules
  \item Open-shells and spin currents
  \item Parallel execution through MPI (optional)
\end{itemize}

GIMIC has so far been interfaced to ACES2 and Turbomole. A small utility
program to extract the AO density and perturbed densities from ACES2
calculations are included in the GIMIC source distribution. Turbomole 5.10 and
newer also has the GIMIC interface built in. 

\section{Installation}
GIMIC is written in pure Fortran 90/95, and thus requires a good F95 compiler
to compile. GIMIC has been compiled and tested to work with the following
compilers: 
\begin{itemize}
  \item GNU gfortran (4.2) 
  \item g95 (0.9)
  \item Intel ifort (please note that ifort 9.0 does not work if optimizations
	are enabled)
  \item Portland pgf90
\end{itemize}

GIMIC uses the standard GNU autoconf generated 'configure' scripts to examine
your system and pick sensible defaults for most build variables. For a
complete list of available configuration options run
\begin{verbatim}
$ ./configure --help
\end{verbatim}

It's recommended that GIMIC is properly installed after compilation, although
not strictly necessary. Here /opt/gimic will be used as the install path. To
configure and build gimic without parallel capabilities run 
\begin{verbatim}
$ ./configure --prefix=/opt/gimic 
$ make install
\end{verbatim}

Then simply add /opt/gimic/bin to your path (or make the appropriate links in
/opt/bin), and you should be up and running. 

If you are not happy with the default compiler picked up by configure you can
override the default by doing
\begin{verbatim}
$ FC=myfavouritef90 ./configure --prefix=/opt/gimic
\end{verbatim}
 
If you want to build the ACES2 interface (xcpdens) you first need to build
ACES2, and then run
\begin{verbatim}
$ ./configure --prefix=/opt/gimic --with-aces2=/path/to/ACES2/lib
\end{verbatim}

If configure can't find the BLAS library (only needed for xcpdens), you need
to specify where to look for it add the following flag to configure: 
\begin{verbatim}
--with-blas-dir=/path/to/lib
\end{verbatim}

If this does not work for some reason you can specify exactly how to link
against BLAS on your system: 
\begin{verbatim}
--with-blas='-L/path/to/my/blas -lfooblas -lwhatever'
\end{verbatim}

Good luck!

\section{Usage}
To run GIMIC three files are needed:
\begin{enumerate}
  \item A file containing the effective one-particle density, and the
	magnetically perturbed densities in AO basis
  \item A MOL file, with information on molecular geometry and basis sets      
  \item A GIMIC input file
\end{enumerate}
The following sections explains how to obtain the density file and the MOL
file using either ACES2 or Turbomole.

\subsection{Running ACES2}
Using ACES2, the special driver script
'\texttt{xgimic2.sh}' must be used to run the NMR shielding calculation. Modify
the script to suit your needs (and set the paths correctly). If the NMR
calculation is done with symmetry, the MOL file must be converted to C1
symmetry using the script \texttt{MOL2mol.sh}, prior to running GIMIC.

Example ZMAT:
\begin{verbatim}
CO2
O    2.14516685791074   0.00000000000000      0.00000000000000
C    0.00000000000622   0.00000000000000      0.00000000000000
O   -2.14516685791393   0.00000000000000      0.00000000000000

*ACES2(CALC=CCSD,BASIS=tzp,UNITS=BOHR
COORD=CARTESIAN
MEMORY=250000000
REFERENCE=RHF
SYMMETRY=ON
PROPERTY=NMR
MULTIPLICITY=1
CHARGE=0
SCF_MAXCYC=200,CC_MAXCYC=150,CC_EXPORDER=40
CC_CONV=10,SCF_CONV=10,LINEQ_CONV=10,CONV=10
LINEQ_EXPAN=30)
\end{verbatim}

Run ACES2 via xgimic2.sh to produce the XDENS file:
\begin{verbatim}
$ xgimic2.sh --cc >aces2.out &
\end{verbatim}

Convert the symmetry adapted MOL file to C1 symmetry: 
\begin{verbatim}
$ MOL2mol.sh
\end{verbatim}
The new MOL file is now called mol.

\subsection{Running Turbomole}
Starting with Turbomole 5.10, the GIMIC interface is part of the official
distribution. To produce the necessary files to run GIMIC, you first need to
optimize the wavefunction/density of the molecule, before running the 
\texttt{mpshift} program to produce the perturbed densities.
Before you run \texttt{mpshift} you need to edit the \texttt{control} file and 
add the \texttt{\$gimic} keyword. When the calculation has finished run the
\texttt{turbo2gimic.py} script (distributed with GIMIC) to produce the
\texttt{mol} and \texttt{XDENS} files.

\subsection{Running GIMIC}
To run gimic you need to have at least three files: The gimic input file
(gimic.inp), the compound density file (XDENS) and the compound basis set and
structure file (mol). Copy the example gimic.inp (in the \texttt{examples/}
directory) to your work directory, edit to your needs, and execute
\begin{verbatim}
$ gimic [--mpi] [gimic.inp] >gimic.out
\end{verbatim}

Before doing the actual calculation it might be a good idea to check that the
grids are correct, run:
\begin{verbatim}
$ gimic --dryrun
\end{verbatim}
and examine the .xyz files that GIMIC produces. If they look ok, simply run
\begin{verbatim}
$ gimic 
\end{verbatim}

If you want to run the parallel version, there is a wrapper script called
'\texttt{qgimic}' (see \texttt{qgimic --help} for a list of command line
options) to produce a generic run script for most queueing systems. Eg. to set
up a parallel calculation with 8 CPUs, 1 h time and 200 MB memory to be run in
\texttt{/work/slask}
\begin{verbatim}
$ qgimic -n 8 -t 01:00 -m 200 /work/slask
\end{verbatim}

This produces a 'gimic.run' file. Edit this file and make sure it's ok, and
then submit it to the queueing system:
\begin{verbatim}
$ qsub gimic.run
\end{verbatim}

\section{The GIMIC input file}
The GIMIC input file is parsed by the getkw input parser, which defines a
grammar based on sections and keywords in a recursive manner. The input
consists of sections containing keywords and/or other sections, and so on. The
input is in principle line oriented, but lines may be continued using a '\\'
at the end of a line. Furthermore, blanks and tabs are insignificant, with the
exception of strings. Lines may be commented until end-of-line with a hash
sign (\#).

Sections are delimited by an opening '{' and closing '}', and may have a
keyword argument enclosed between '(' and ')'.

Keywords come in two different types; simple keywords consisting of integers,
reals or strings (enclosed in `` ''), and array keywords. Array keywords are
enclosed in '[' ']' and elements -- integers, reals or strings -- are delimited 
by ','.

\subsection{Keywords}
The top level section defines a few global parameters:
\begin{description}
  \item[dryrun=off] Don't actually calculate anything. 
	Good for tuning grids, etc. Can also be specified on the command line.
  \item[mpirun=off] [boolean] Run in parallel mode.
  \item[title] Useless keyword, but since every program with a bit of self
	respect has a title, GIMIC also has one\ldots
  \item[basis=mol] Name of the MOL file (eg. MOL or mol or whatever)	
  \item[density=XDENS] Name of the density file (eg. XDENS)	
  \item[spherical=off] Use spherical cartesians
	(i.e. 5d/7f/10g\ldots). This is usually handled automagically. Experts only.
  \item[debug=1] Set debug level. The higher the number, the more useless
	output one gets.
  \item[diamag=on] Turn on/off diamagnetic contributions
  \item[paramag=on]  Turn on/off paramagnetic contributions
  \item[openshell=false] Open-shell calculation
  \item[screening=off]    Use screening to speed up calculations
  \item[screen\_thrs=1.d-8]  Screening threshold
  \item[show\_up\_axis=true]   Mark the "up" axis in .xyz files
  \item[calc=\lbrack cdens,\ldots\rbrack]  This keyword determines what is to be
	calculated, and in what order. Possible options are: 'cdens' -- calculate
	current densities, 'integrate' -- integrate the current flow through a
	cut-plane, 'divj' -- calculate the divergence of the current. Each of
	these options have their own respective sections to specify options and
	grids.
\end{description}

\subsection{The current density}
\subsubsection*{Section: cdens}
\begin{description}
  \item[jtensor=JTENSOR] Name of output file  containing the current tensors
  \item[jvector=JVECTOR] Name of output file  containing the current vectors
  \item[magnet=\rbrack 0.0, -1.0, 0.0\lbrack] Vector which specifies the 
	direction of the magnetic field. 
\item magnet\_axis=z] Specify the magnetic field along a defined axis. Valid
  options are: i,j,k or x,y,z or T. ``i,j,k'' are the directions of the basis
  vectors defining the computational grid after any Euler rotation. ``x,y,z''
  are the absolute fixed laboratory axis. ``T`` is used for integration and
  specifies the direction which is orthogonal to the molecular plane, but
  parallel to the integration plane.
  \item[scale\_vectors=1.0] Scaling factor for plotting purposes.
  \item[diamag=on] Annihilate the diamagnetic contribution to the
	current. Experts only.
  \item[paramag=on] Annihilate the paramagnetic contribution to the
	current. Experts only.
  \item[grid(std)] [subsection] Grid to be used for calculating the currents. 
	See the ''Grids`` section for a description of how to specify grids.
  \item[plot(on)] [subsection] Produce files suitable for plotting with
	'gnuplot' or 'gopenmol'
	\begin{description}
	  \item[vector=JVEC]  File to contain the current vector field (gnuplot
		friendly)	
	  \item[modulus=JMOD]  File to contain the modulus of the current density 
		(gnuplot friendly)
	  \item[nvector=NJVEC]  File to contain the normalized current vector field 
		(gnuplot friendly). Mostly useful for debugging purposes.
	  \item[gopenmol=jmod.plt]  File to contain the current density in a
		gopenmol friendly format.
	\end{description}
\end{description}

\subsection{Integration}
\subsubsection*{Section: integral}
\begin{description}
\item magnet\_axis=T] Specify the magnetic field along 
  the direction which is orthogonal to the molecular plane, but
  parallel to the integration plane.
  \item[magnet=\rbrack 0.0, -1.0, 0.0\lbrack] Vector which specifies the 
	direction of the magnetic field. 
  \item[modulus=off]  Calculate the mod(J) integral, this is useful to verify
	that the actual integration grid is sensible in ``tricky'' molecules.
  \item[tensor=off]  Integrate the tensor components
  \item[interpolate=off] If a calculation has been preformed on a even
	spaced grid, generate a grid suitable for Gaussian integration by doing
	Lagrange interpolation
  \item[lip\_order=5] Polynomial order of the Lagrange Interpolation
	Polynomials
  \item[grid(bond)] [subsection] Grid to be used for calculating the currents. 
	See the ''Grids`` section for a description of how to specify grids.
\end{description}

\subsection{The divergence of the current field}
\subsubsection*{Subsection: divj}
\begin{description}
  \item[magnet=\rbrack 0.0, -1.0, 0.0\lbrack] Vector which specifies the 
	direction of the magnetic field. 
\item magnet\_axis=z] Specify the magnetic field along a defined axis. Valid
  options are: i,j,k or x,y,z or T. ``i,j,k'' are the directions of the basis
  vectors defining the computational grid after any Euler rotation. ``x,y,z''
  are the absolute fixed laboratory axis. ``T`` is used for integration and
  specifies the direction which is orthogonal to the molecular plane, but
  parallel to the integration plane.
  \item[gopemol=divj.plt] Filename of gOpenMol plot
  \item[grid(std)] [subsection] Grid to be used for calculating the currents. 
	See the ''Grids`` section for a description of how to specify grids.
\end{description}

\subsection{The electronic density}
The GIMIC program can also produce plots of the electronic density. This code
is very rudimentary currently, and cannot produce densities of specific MOs or
ranges of MOs.

\subsubsection*{Section: edens}
\begin{description}
  \item[density='EDENS'] Filename which contains AO density. XDENS is fine
	usually.
  \item[density\_plot='edens\_plt.txt'] File name of density plot
  \item[gopemol=edens.plt] Filename of gOpenMol plot
  \item[grid(std)] [subsection] Grid to be used for calculating the currents. 
	See the ''Grids`` section for a description of how to specify grids.
\end{description}

\section{Grids}
There are two principal types of grids; the simple 'std (or base)' grid, which
is defined by a pair of (orthogonal) basis vectors, and the 'bond' grid which
is mostly useful for defining cut-planes through bonds for integration. There
is also a third grid type 'file', which specifies a file containing gridpoints
For the exact format of this file please refer to the source in grid.f90.
Furthermore there are two types of grids, evenly spaced or with grid points
distributed for 
Gauss-Legendere or Guass-Lobato quadrature. This is specified with the
'type=even|gauss|lobato' keyword. When a quadrature grid is specified the
order of the quadrature must also be specified with the 'gauss\_order'
keyword.
The number of grid points in each direction is specified either explicitly
using either of the array keywords 'grid\_points' or 'spacing'. If the chosen
grid is not a simple even spaced grid, the actual number of grid points will
be adjusted upwards to fit the requirements of the chosen quadrature.

The shape of the grid can also be modified by the 'radius' key, which
specifies a cutoff radius. This can be useful for integration.
Sometimes it's practical to be able to specify a grid relative to a well know
starting point. The 'rotation' keyword specifies Euler angles for rotation
according to the x->y->z convention. Note that the magnetic field is not
rotated, unless it is specified with 'magnet\_axis=i,j or k'.

GIMIC automatically output a number of .xyz files containing dummy points to
show how the grids defined actually are laid out in space. 

\subsection{Basic grids}
The 'std' grid is defined by giving an 'origin'
and two orthogonal basis vectors 'ivec' and 'jvec' which define a plane. The
third axis is determined from $\vec k=\vec i\times\vec j$.
The array 'lengths' specifies the grid dimensions in each direction.

\subsubsection*{grid(std):}
\begin{description}
  \item[type=even] 
  \item[origin=\lbrack -8.0, -8.0, 0.0\rbrack]  Origin of grid
  \item[ivec=\lbrack 1.0, 0.0, 0.0\rbrack]        Basis vector i
  \item[jvec=\lbrack 0.0, 1.0, 0.0\rbrack]       Basis vector j ( k = i x j )
  \item[lengths=\lbrack 16.0, 16.0, 0.0\rbrack]     Lenthts of (i,j,k)
  \item[spacing=\lbrack 0.5, 0.5, 0.5\rbrack]  Spacing of points on grid (i,j,k)
  \item[grid\_points=\lbrack 50, 50, 0\rbrack]    Number of gridpoints on grid (i,j,k)
  \item[rotation=\lbrack 0.0, 0.0, 0.0\rbrack]   Rotation of (i,j,k) -> (i',j',k') in
	degrees. Given as Euler angles in the x->y->z convention.
\end{description}

\subsection{Bond grids}
The 'bond' type grids define a plane through a bond, or any other defined
vector. The plane is orthogonal to the vector defining the bond. The bond can
be specified either by giving two atom indices, 'bond=[1,2]', or by specifying
a pair of coordinates, 'coord1' and 'coord2'. The position of the grid between
two atoms is determined by the 'distance' key, which specifies the distance
from atom 1 towards atom 2.
For analysing dia- and
paramagnetic contributions, the positive direction of the bond is taken to be
from atom 1 towards atom 2. Since one vector is not enough to uniquely
defining the coordinate system (rotations around the bond are arbitrary), a
fixpoint must be specified using either the 'fixpoint' atom index or 
the 'fixcoord' keyword. This triple of coordinates is also used to fix the
direction of the magnetic field when the 'magnet\_axis=T' is used.

The shape and size of the bond grid can be specified in two ways. The origin
of the grid is fixed to the center of the bond, and all specifications are
relative to this origin. The first method specifies lengths in four directions
(they can be negative, as long as they pairwise sum up to a positive number):
\begin{verbatim}
up=5.0
down=5.0
in=1.5
out=5.0
\end{verbatim}
The other way to specify the shape is using the 'height' and 'width' keywords,
which specify intervals relative to the origin:
\begin{verbatim}
height=[-5.0, 5.0]
width=[-1.5, 5.0]
\end{verbatim}

\subsubsection*{grid(std):}
\begin{description}
  \item[type=gauss|lobato]  Use uneven distribution of grid points for
	quadrature
  \item[bond=\lbrack1,2\rbrack] Atom indices for bond specification
  \item[fixpoint=5] Atom index to use for fixing the magnetic field and grid
	orientation
  \item[coord1=\lbrack 0.0, 0.0, 3.14\rbrack] Coordinate of atom 1
  \item[coord2=\lbrack 0.0, 0.0, -3.14\rbrack] Coordinate of atom 2
  \item[fixcoord=\lbrack 0.0, 0.0, 0.0\rbrack] Fixation coordinate 
  \item[distance=1.5]  Place grid 'distance' between atoms 1 and atom 2
  \item[gauss\_order=9] Order for Gauss quadrature
\item[spacing=\lbrack 0.5, 0.5, 0.5\rbrack]  Spacing of points on grid (i,j,k)
  (approximate)
\item[grid\_points=\lbrack 50, 50, 0\rbrack]   Number of grid points on grid 
  (i,j,k) (approximate)
\item[up=4.0]  Grid size in $\vec i$ direction
\item[down=4.0] Grid size in $-\vec i$ direction
\item[in=1.0]  Grid size in $-\vec j$ direction
\item[out=6.0] Grid size in $\vec j$ direction
\item[height=\lbrack -4.0, 4.0\rbrack] Grid size relative to grid center
\item[width=\lbrack -1.0, 6.0\rbrack] Grid size relative to grid center
\item[radius=3.0]  Create a round grid by cutting off at radius 
\item[rotation=\lbrack 0.0, 0.0, 0.0\rbrack]   Rotation of (i,j,k) ->
  (i',j',k') in degrees. Given as Euler angles in the x->y->z convention.
\end{description}

\end{document}
