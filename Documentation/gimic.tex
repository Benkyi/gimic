\documentclass[a4paper,11pt]{article}
\usepackage[english]{babel}
\usepackage[T1]{fontenc}
\usepackage[utf8]{inputenc}
%\usepackage{hyperref}
\usepackage{times}
\usepackage{mystuff}
\usepackage{mytitlepage}
%\usepackage[pdftex,colorlinks=true,
%   pdfstartview=FitV,
%   linkcolor=blue,
%   citecolor=blue,
%   urlcolor=blue]{hyperref}

\begin{document}
\title{The GIMIC program}
%\title{Gauge-Including Magnetically Iduced Currents Program}

\subtitle{User manual (v. 0.1)}
\author{{\sf Jonas Jus\'elius}}
\address{
{\sf University of Tromsø}\\
{\sf Department of Chemistry}\\
{\sf N-9037 Tromsø}
}
\years{2004,2005}
\abstract{
\centerline{\Huge\bf This manual is obsolete!!!}}
\maketitle

\section{Introduction}
The GIMIC program is program to calculate magnetically induced current
densities in molecules. The following features have been implemented in the
program
\begin{itemize}
  \item Current densities in 2D or 3D
  \item The modulus of the current 	
  \item The divergence of the current (this is useful for checking gauge
	invariance vs. gauge independence)
  \item Vector representation of the current in 2D or 3D
  \item Integration of the current flow through defined cut-planes in
	molecules
  \item Parallel execution through MPI (optional)
\end{itemize}

GIMIC has so far been interfaced to ACES2 and Turbomole. A small utility
program to extract the AO density and perturbed densities from ACES2
calculations are included in the GIMIC source distribution. The Turbomole
interface is rather primitive (only HF/SCF and DFT densities), and
unfortunately still requires a specially patched version of Turbomole.

\section{Installation}
GIMIC is written in pure Fortran 90/95, and thus requires a good F95 compiler
to compile. At the moment of writing GNU gfortran (4.0.2) is not able to
compile GIMIC, partly due to missing language features and partly due to
an incomplete standard f95 library.

GIMIC is dependent on the library {\tt libgetkw} to handle input file parsing.
Both libgetkw and GIMIC rely on GNU autoconf generated {\tt configure} scripts
to
set up the system dependent build parameters. If the ACES2 interface is to be
built, one needs to ensure the availability of {\tt liblibr.a} from the ACES2
distribution.

The \texttt{configure} scripts shipped with libgetkw, GIMIC and ACES2 are
quite flexible, and for a full list of options run
\begin{verbatim}
$ ./configure --help
\end{verbatim}
for each of the programs. 

The recommended procedure to build GIMIC is to first build libgetkw and
install it. libgetkw can conveniently be installed in its build directory,
e.g. (note the back quotes!)
\begin{verbatim}
$ ./configure --prefix=`pwd`
$ make install
\end{verbatim}

To build GIMIC
\begin{verbatim}
$ ./configure --prefix=/usr/local --with-getkwdir=/path/to/libgetkw 
$ make install
\end{verbatim}
or optionally if the ACES2 interface is required
\begin{verbatim}
$ ./configure --prefix=/usr/local --with-getkwdir=/path/to/libgetkw \
--with-aces2=/path/to/aces2/lib
\end{verbatim}
Furthermore, if BLAS and LAPACK are installed in non-standard locations you
might need to specify their locations (\texttt{./configure --help} is your
friend).

\section{Usage}
GIMIC needs three input files to calculate the current density:
\begin{enumerate}
  \item A file containing the effective one-particle density, and the
	magnetically perturbed densities in AO basis
  \item A MOL file, with information on molecular geometry and basis sets	
  \item A GIMIC input file
\end{enumerate}

Currently the density file (usually called XDENS) can be obtained using either 
ACES2 or [a special version of] Turbomole. Using ACES2, the special driver
script 'xgimic2.sh' must be used to run the NMR shielding calculation. Modify
the script to suit your needs (and set the paths correctly). If the NMR
calculation is done with symmetry, the MOL file must be converted to C1
symmetry using the script MOL2mol.sh, prior to running GIMIC. Similarly there
is a script turbo2mol.sh for converting Turbomole control/coord/basis into a
MOL file for GIMIC.

Example ZMAT:
\begin{verbatim}
CO2
O    2.14516685791074   0.00000000000000      0.00000000000000
C    0.00000000000622   0.00000000000000      0.00000000000000
O   -2.14516685791393   0.00000000000000      0.00000000000000

*ACES2(CALC=CCSD,BASIS=tzp,UNITS=BOHR
COORD=CARTESIAN
MEMORY=250000000
REFERENCE=RHF
SYMMETRY=ON
PROPERTY=NMR
MULTIPLICITY=1
CHARGE=0
SCF_MAXCYC=200,CC_MAXCYC=150,CC_EXPORDER=40
CC_CONV=10,SCF_CONV=10,LINEQ_CONV=10,CONV=10
LINEQ_EXPAN=30)

\end{verbatim}

Run ACES2 via xgimic2.sh to produce the XDENS file:
\begin{verbatim}
$ xgimic2.sh --cc >aces2.out &
\end{verbatim}

Convert the symmetry adapted MOL file to C1 symmetry: 
\begin{verbatim}
$ MOL2mol.sh
\end{verbatim}
The new MOL file is now called mol.

Create a GIMIC input file, and run GIMIC
\begin{verbatim}
$ gimic [--mpi] [gimic.inp] >gimic.out
\end{verbatim}

\section{The GIMIC input file}
The GIMIC input file is parsed by libgetkw, which defines a parser
based on sections and keywords in a recursive manner. The input consists of
sections containing keywords and/or other sections, and so on. The input is in
principle line oriented, but lines may be continued using a '\\' at the end of a
line. Furthermore, blanks and tabs are insignificant, with the exception of
strings. Lines may be commented until end-of-line with a hash sign (\#).

Sections are delimited by an opening '{' and closing '}', and may have a keyword
argument enclosed between '(' and ')'.

Keywords come in two different types; simple keywords consisting of integers,
reals or strings (enclosed in `` ''), and array keywords. Array keywords are
enclosed in '[' ']' and elements -- integers, reals or strings -- are delimited 
by ','.

\subsection{Keywords}
The top level section is CDENS. This section defines a few global parameters:
\begin{description}
  \item[title] [str] Useless keyword, but since every program with a bit of self
	respect has a title, GIMIC also has one\ldots
  \item[rerun] [boolean] Experts only.
  \item[basis] [str] Name of the MOL file (eg. MOL or mol or whatever)	
  \item[density] [str] Name of the density file (eg. XDENS)	
  \item[spherical] [boolean] Use spherical cartesians (i.e. 5d/7f/10g\ldots).
	This is usually handled automagically. Experts only.
  \item[debug] [int] Set debug level. The higher the number, the more useless
	output one gets.
  \item[calc] [array(str)] This keyword determines what is to be calculated, and
	in what order. Possible options are: 'cdens' -- calculate current densities,
	'integrate' -- integrate the current flow through a cut-plane, 'divj' --
	calculate the divergence of the current. Each of these options have their
	own respective sections to specify options and grids.
\end{description}

\subsubsection{cdens}
\begin{description}
  \item[jtensor] [str] Name of output file  containing the current tensors
	(optional)
  \item[jvector] [str] Name of output file  containing the current vectors
	(optional)
  \item[orthogonal\_magnet] [boolean] Automatically make magnetic field
	perpendicular to the computational grid
  \item[magnet] [array(3*real)] Vector which specifies the direction of the
	magnetic field. 
  \item[scale\_vectors] [real] Scaling factor for plotting purposes.
  \item[diamag] [boolean] Annihilate the diamagnetic contribution to the
	current. Experts only.
  \item[paramag] [boolean] Annihilate the paramagnetic contribution to the
	current. Experts only.
  \item[plot([boolean])] [subsection] Produce files suitable for plotting with
	'gnuplot' or 'gopenmol'
	\begin{description}
	  \item[plots] [int] Number of plots when doing 3D calculations. Best to set
		equal to 1, and ignore altogether.
	  \item[vector] [str] File to contain the current vector field (gnuplot
		friendly)	
	  \item[modulus] [str] File to contain the modulus of the current density 
		(gnuplot friendly)
	  \item[nvector] [str] File to contain the normalized current vector field 
		(gnuplot friendly). Mostly useful for debugging purposes.
	  \item[gopenmol] [str] File to contain the current density in a gopenmol
		friendly format.
	\end{description}
  \item[grid] [subsection] Grid to be used for calculating the currents. See
	below for a description of the 'grid' section.
\end{description}

\subsubsection{Grids}
Grid definition in GIMIC is a tad bit messy currently. There are a number of
different ways to define grids (most of them pretty experimental), but for
simple plots it's (fortunately) pretty straight forward. 

There are two principal types of grids; the simple 'std' grid, which is
defined by a pair of (orthogonal) basis vectors, and the 'bond' grid which is
mostly useful for defining cut-planes for integration.

The 'std' grid is defined by giving an origin where the grid should start,
and then two vectors 'v1' and 'v2' which define a plane. The length of the
vectors v1 and v2 determine the size of the grid plane. The two vectors
implicitly define a third vector 'v3', orthogonal to v1 and v2, and which
length is given by the 'l3' parameter. Thus, for 3D grids l3 is non-zero.
The number (the spacing of points) of grid points in the directions v1, v2, 
and v3 is given by the 'step' vector.

The 'bond' grid is defined in the following way: v1 is the coordinate of atom
1 and v2 is the coordinate of atom 2, and l3 is the distance from atom 1 to
the grid (usually one wants l3 to be half of the bond distance between atom 1
and 2). The grid will be orthogonal to the vector between atoms 1 and 2, and
it's size is determined by the 'height' and 'width' parameters, which are 2
dimensional vectors to be able to specify unsymmetrical grids. The 'origin'
parameter is only used to fix the ``torsional angle'' (or what ever one wants
to call it).

GIMIC automatically output a number of .xyz files containing dummy points to
show how the grids defined actually are laid out in space. The 'gridplot' 
parameter gives the (approximate) number of dummy points to use in each
direction.

\textbf{grid(type)} [(std|bond)]
\begin{description}
  \item[type] [str] (even|gauss) Determines the distribution of grid points;
	even spaced or suitable for Gaussian quadrature
  \item[origin] [array(3*real)]
  \item[v1] [array(3*real)]	
  \item[v2] [array(3*real)]	
  \item[l3] [real]
  \item[height] [array(2*real)]
  \item[width] [array(2*real)]
  \item[step] [array(3*real)] Spacing between grid points in x,y,z directions 
  \item[map] [array(2*real)] Origin for the plotable data in 2D space
  \item[gridplot] [int] Approximately the number of points in each direction to
	be used in the .xyz file which shows the grid.
\end{description}

\subsection{Integration}

\begin{description}
  \item[interpolate] [boolean] If a calculation has been preformed on a even
	spaced grid, generate a grid suitable for Gaussian integration by doing
	Lagrange interpolation
  \item[lip\_order] [int] Polynomial order of the Lagrange Interpolation
	Polynomials
  \item[grid(std|bond)] 
\end{description}

\subsection{The divergence}

\begin{description}
  \item[plots] [array(int)] produce plots for v3 indices defined by the vector
  \item[gopemol] [str] filename of gOpenMol plot
  \item[grid(std|bond)] 
\end{description}
\end{document}
